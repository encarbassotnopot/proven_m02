\documentclass[a4paper,12pt]{article}
\usepackage[utf8]{inputenc}

\usepackage[spanish]{babel} % Patrons de trencament de paraula
\usepackage{fancyhdr} % Per la capçalera
\usepackage{geometry}
\usepackage{graphicx} % Per importar logos (i altres gràfics)
\usepackage[colorlinks,linkcolor=black]{hyperref} % Per fer que l'index tingui hiperlinks en el pdf
\usepackage{indentfirst}
\usepackage[sc,small]{titlesec} % Seccions personalitzades

%% Títols
\newcommand{\modulnum}{2}
\newcommand{\modulnom}{Bases de Dades}

\newcommand{\ufnum}{2}
\newcommand{\ufnom}{Llenguatges SQL: DML i DDL}

\newcommand{\acttipus}{Pràctica 1 - Fita 3}
\newcommand{\actnom}{SQL Mistery - Construir la base de dades}

%% Entreliniats
\linespread{1.5}

%% Capçalera
\pagestyle{fancy}
\setlength{\headheight}{40pt}
\addtolength{\topmargin}{-20pt}
\fancyhead[L]{\includegraphics*[height=35pt]{provencana_bw.pdf}}
\fancyhead[R]{
	{\scshape\scriptsize Mòdul \modulnum: \modulnom}\\
	{\scshape\footnotesize \acttipus \space  - \actnom}\\
	{\scshape\small Eina Coma Bages}}

\addtolength{\textheight}{2cm}

%% Comandes personalitzades
\newcommand{\mygraphic}[2][width=\textwidth]{\begin{center}
		\centering\includegraphics[#1]{#2}\par
\end{center}}

\renewcommand{\contentsname}{Índex}

\begin{document}
\include{portada.tex}

\section{Càrrega i execució de l'script}
Premem upload i s'obre una finestra emergent per carregar el nostre script.
\mygraphic{imatges/1.png}

Seleccionem el fitxer i el pengem.
\mygraphic{imatges/2.png}

\newpage
Una vegada penjat, el podem executar.
\mygraphic{imatges/3.png}

Ens adverteix del que volem fer, i si efectivament volem executar-lo, premem el botó.
\mygraphic{imatges/4.png}

\newpage
La primera execució acaba amb deu errors, causats pels drops de taules que encara no existeixen.
\mygraphic{imatges/5.png}

Si tornem a la pantalla principal i repetim el mateix procediment, veurem que les execucions posteriors acaben sense cap error.
\mygraphic{imatges/6.png}
\mygraphic{imatges/7.png}
\mygraphic{imatges/8.png}

\section{Comprovació des de ll'Object Browser}
\mygraphic{imatges/9.png}
\mygraphic{imatges/10.png}

\section{Comprovació amb Queries}
\mygraphic{imatges/11.png}
\mygraphic{imatges/12.png}

\end{document}