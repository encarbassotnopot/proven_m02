\documentclass[a4paper,12pt]{article}
\usepackage[utf8]{inputenc}

\usepackage[spanish]{babel} % Patrons de trencament de paraula
\usepackage{fancyhdr} % Per la capçalera
\usepackage{geometry}
\usepackage{graphicx} % Per importar logos (i altres gràfics)
\usepackage[colorlinks,linkcolor=black]{hyperref} % Per fer que l'index tingui hiperlinks en el pdf
\usepackage{indentfirst}
\usepackage[sc,small]{titlesec} % Seccions personalitzades

%% Títols
\newcommand{\modulnum}{2}
\newcommand{\modulnom}{Bases de Dades}

\newcommand{\ufnum}{2}
\newcommand{\ufnom}{Llenguatges SQL: DML i DDL}

\newcommand{\acttipus}{Juguem al SQL Mistery}
\newcommand{\actnom}{Olympus}

\newcommand{\autor}{Eina Coma Bages}

%% Entreliniats
\linespread{1.5}

%% Capçalera
\pagestyle{fancy}
\setlength{\headheight}{40pt}
\addtolength{\topmargin}{-20pt}
\fancyhead[L]{\includegraphics*[height=35pt]{provencana_bw.pdf}}
\fancyhead[R]{
	{\scshape\scriptsize Mòdul \modulnum: \modulnom}\\
	{\scshape\footnotesize \acttipus \space  - \actnom}\\
	{\scshape\small\autor}}

\addtolength{\textheight}{2cm}

%% Comandes personalitzades
\newcommand{\mygraphic}[2][width=\textwidth]{\begin{center}
		\centering\includegraphics[#1]{#2}\par
\end{center}}

\renewcommand{\contentsname}{Índex}

\begin{document}
\begin{titlepage}
	\centering
	\includegraphics*[width=0.15\textwidth]{provencana_color.pdf}
	\par\vspace{0.5cm}

	{\scshape\Large Institut Provençana \par}

	\vspace{1cm}

	{\itshape\Large \acttipus \par}
	{\bfseries\LARGE \actnom \par}
	
	\vspace{1cm}
	
	{\scshape\large Mòdul \modulnum: \par}
	{\scshape\Large \modulnom \par}

	\vspace{1cm}
	
	{\scshape\normalsize Unitat Formativa \ufnum: \par}
	{\scshape\large \ufnom \par}

	\vfill
	{\Large\itshape Eina Coma Bages \\ Iris Hidalgo Palomino\par}
	\vfill

	Curs 2022/2023
\end{titlepage}
\section{Part inicial}
\textsc{1. Quin és el teu nom i cognoms?}

Eina Coma Bages

\textsc{2. Quin és el nom del grup del SQL Mistery al que jugues?}

Olympus 

\textsc{3. Quin és el punt de partida:}
\begin{itemize}
	\item \textsc{Què et demanen resoldre? Quina és la pregunta que has de respondre?}
	
	Qui ha regalat un número de loteria a cada casa del poble?	

	\item \textsc{Quina informació/dada inicial et donen per començar la investigació?}

	En un poble de 150 habitants algú ha deixat un número de loteria a cada casa. La notícia és d'una setmana abans del sorteig del sorteig del "Gordo de Navidad", celebrat el 22 de desembre del 2002.

\end{itemize}

\newpage
\section{Part intermitja}

\subsection{Primera Consulta}

\subsubsection{Consulta SQL}
\mygraphic{imatges/1.png}

\subsubsection{Dades obtingudes}
\mygraphic{imatges/2.png}

\subsubsection{Interpretació de les dades}
Revisem la notícia per veure si podem treure més informació, però no ha estat així.

\subsubsection{Següent pas}
Passem a buscar quins pobles tenen 150 habitants.

\newpage
\subsection{Segona Consulta}

\subsubsection{Consulta SQL}
\mygraphic{imatges/3.png}

\subsubsection{Dades obtingudes}
\mygraphic{imatges/4.png}

\subsubsection{Interpretació de les dades}
Hi ha un total de sis pobles amb 150 habitants.

\subsubsection{Següent pas}
Intentem esbrinar a quin poble ha passat mirant les entrevistes.

\newpage
\subsection{Tercera Consulta}

\subsubsection{Consulta SQL}
\mygraphic{imatges/5.png}

\subsubsection{Dades obtingudes}
\mygraphic{imatges/6.png}

\subsubsection{Interpretació de les dades}
Comprovem que el poble on ha tocat, és Wein. El número guanyador és el 17421.

També algú diu que el "misteriós Pare Noel" viu a la primera casa.

\subsubsection{Següent pas}
Podem fer dues coses, consultar qui va vendre el número o trucar a la porta de la casa. Començem amb la segona opció

\newpage
\subsection{Quarta Consulta}

\subsubsection{Consulta SQL}
\mygraphic{imatges/7.png}

\subsubsection{Dades obtingudes}
\mygraphic{imatges/8.png}

\subsubsection{Interpretació de les dades}
Hi viu una persona amb l'id 228.

\subsubsection{Següent pas}
Busquem si ha dit alguna cosa.

\newpage
\subsection{Cinquena Consulta}

\subsubsection{Consulta SQL}
\mygraphic{imatges/9.png}

\subsubsection{Dades obtingudes}
\mygraphic{imatges/10.png}

\subsubsection{Interpretació de les dades}
Sabem que no ha estat cosa d'aquesta persona, però potser el seu fill hi té alguna cosa a veure.

\subsubsection{Següent pas}
Tornem enrere per consultar qui va vendre el número afortunat.

\newpage
\subsection{Sisena Consulta}

\subsubsection{Consulta SQL}
\mygraphic{imatges/11.png}

\subsubsection{Dades obtingudes}
\mygraphic{imatges/12.png}

\subsubsection{Interpretació de les dades}
L'ha venut un tal Carter Verdon, amb l'id de 5.

\subsubsection{Següent pas}
Consultem si té alguna entrevista.

\newpage
\subsection{Setena consulta}

\subsubsection{Consulta SQL}
\mygraphic{imatges/13.png}

\subsubsection{Dades obtingudes}
\mygraphic{imatges/14.png}

\subsubsection{Interpretació de les dades}
El va vendre a un home amb un nom que començava per D, estava casat i tenia entre 30 i 40 anys.

\subsubsection{Següent pas}
Busquem si algú encaixa en aquest perfil.

\newpage
\subsection{Vuitena Consulta}

\subsubsection{Consulta SQL}
\mygraphic{imatges/15.png}

\subsubsection{Dades obtingudes}
\mygraphic{imatges/16.png}

\subsubsection{Interpretació de les dades}
Tenim dos sospitosos, però el primer ja l'havíem descartat anteriorment. Queda només en David Ring.

\subsubsection{Següent pas}
Per comprovar si hem trobat al Pare Noel de Wein, fem un join amb les entrevistes.

\newpage
\subsection{Novena Consulta}

\subsubsection{Consulta SQL}
\mygraphic{imatges/17.png}

\subsubsection{Dades obtingudes}
\mygraphic{imatges/18.png}

\subsubsection{Interpretació de les dades}
Com confirma en l'entrevista, en David Ring és qui va regalar un dècim a totes les cases de Wein.

\newpage
\section{Part final}

\textsc{6. Has aconseguit resoldre el SQL Mystery (Sí/No)? Quina és la teva solució final al Mystery?}

L'he resolt, qui ha regalat un número de loteria a totes les cases ha estat en David Ring.

\textsc{7. En una escala de l’1 al 10, t’ho has passat bé? Ha estat divertit?}

10, ha estat molt divertit.

\textsc{8. En una escala de l’1 al 10, quan has après resolent aquest SQL Mystery?}

1

\end{document}