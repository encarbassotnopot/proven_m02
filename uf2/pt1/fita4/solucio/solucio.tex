\documentclass[a4paper,12pt]{article}
\usepackage[utf8]{inputenc}

\usepackage[spanish]{babel} % Patrons de trencament de paraula
\usepackage{fancyhdr} % Per la capçalera
\usepackage{geometry}
\usepackage{graphicx} % Per importar logos (i altres gràfics)
\usepackage[colorlinks,linkcolor=black]{hyperref} % Per fer que l'index tingui hiperlinks en el pdf
\usepackage{indentfirst}
\usepackage[sc,small]{titlesec} % Seccions personalitzades

%% Títols
\newcommand{\modulnum}{2}
\newcommand{\modulnom}{Bases de Dades}

\newcommand{\ufnum}{2}
\newcommand{\ufnom}{Llenguatges SQL: DML i DDL}

\newcommand{\acttipus}{Pràctica 1 - Fita 4}
\newcommand{\actnom}{Solució - La Caixa Forta de la Mort}

\newcommand{\autor}{Eina Coma Bages}


%% Entreliniats
\linespread{1.5}

%% Capçalera
\pagestyle{fancy}
\setlength{\headheight}{40pt}
\addtolength{\topmargin}{-20pt}
\fancyhead[L]{\includegraphics*[height=35pt]{provencana_bw.pdf}}
\fancyhead[R]{
	{\scshape\scriptsize Mòdul \modulnum: \modulnom}\\
	{\scshape\footnotesize \acttipus \space  - \actnom}\\
	{\scshape\small\autor}}

\addtolength{\textheight}{2cm}

%% Comandes personalitzades
\newcommand{\mygraphic}[2][width=\textwidth]{\begin{center}
		\centering\includegraphics[#1]{#2}\par
\end{center}}

\renewcommand{\contentsname}{Índex}

\begin{document}
\include{portada.tex}
\tableofcontents

\newpage
\section{Informació inicial}
Cada divendres a les 15h es revisa el contingut de la caixa forta i el robatori es va descobrir el divendres 18/03/2022.

D'aquí deduïm que el robatori s'ha dut a terme entre el divendres anterior (11/03) i aquest (18/03)

\newpage
\section{Primer pas}
\subsection{Query}
\mygraphic{imatges/1a.png}
\subsection{Resultats}
\mygraphic{imatges/1b.png}
\subsection{Dades obtingudes}
Hi ha una entrada sospitosa la matinada del 16 de març. Sembla que algú ha esborrat el vídeo de la càmera de vigilància.
\subsection{Següent pas}
Hem de trobar qui va ser l'última persona que ha accedit a l'edifici abans del robatori. Aquesta informació la podem trobar a la taula d'accessos.

\newpage
\section{Segon pas}
\subsection{Query}
\mygraphic{imatges/2a.png}
\subsection{Resultats}
\mygraphic{imatges/2b.png}
\subsection{Dades obtingudes}
Ja sabem amb quina clau es va entrar abans del robatori.
\subsection{Següent pas}
Ara hem de saber de qui era aquesta clau, amb la taula poseeix.

\newpage
\section{Tercer pas}
\subsection{Query}
\mygraphic{imatges/3a.png}
\subsection{Resultats}
\mygraphic{imatges/3b.png}
\subsection{Dades obtingudes}
La clau era de la Tamara, amb l'id 18, però va ser desactivada poc després.
\subsection{Següent pas}
Investiguem si la Tamara va tenir cap problema amb la clau. Ho farem buscant a la taula dels tiquets.

\newpage
\section{Quart pas}
\subsection{Query}
\mygraphic{imatges/4a.png}
\subsection{Resultats}
\mygraphic{imatges/4b.png}
\subsection{Dades obtingudes}
Veiem els tiquets que ha obert la Tamara. Va perdre la clau abans del robatori. Uns dies més tard, l'Oliver la va trobar en un Seat blau.
\subsection{Següent pas}
Hem d'esbrinar qui va agafar aquest cotxe abans que ho fes l'Oliver. Aquesta persona és qui va prendre la clau a la Tamara i després se la va deixar allà.

\newpage
\section{Cinquè pas}
\subsection{Query}
\mygraphic{imatges/5a.png}
\subsection{Resultats}
\mygraphic{imatges/5b.png}
\subsection{Dades obtingudes}
Abans però, obtenim l'id de l'Oliver.
\subsection{Següent pas}
Amb aquesta dada, ja podem buscar qui va agafar el Seat blau abans que ho fes ell.

\newpage
\section{Sisè pas - Solució}
\subsection{Query}
\mygraphic{imatges/6a.png}
\subsection{Resultats}
\mygraphic{imatges/6b.png}
\subsection{Dades obtingudes}
I ja hem resolt el misteri. L'empleat que va robar la clau de la Tamara, va emportar-se els diners de la caixa forta i es va deixar la clau al cotxe va ser en Preston Harding.

\newpage
\section{Aprenentatges}
Amb aquesta pràctica he:
\begin{enumerate}
	\item Après a generar dades a l'engròs,
	\item Consolidat els coneixements de DDL i
	\item Vist com es crea una base de dades, partint d'un disseny conceptual amb el diagrama ER, aplicant la transformació relacional i finalment convertint això a SQL.
\end{enumerate}

\end{document}